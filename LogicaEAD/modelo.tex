\documentclass[12pt]{article}
\usepackage[latin1]{inputenc}
\usepackage{times}
\usepackage{enumerate}
\usepackage{fullpage}
\usepackage{graphicx}
\usepackage{naturaldeduction}
\usepackage{color}

\begin{document}

\noindent
Universidade Federal do Rio Grande do Sul \hfill Instituto de Inform�tica \newline 
INF05508 -- L�gica para Computa��o \hfill 2018/2 \newline
Aluno \hfill Jo�o Pedro Porto Pires de Oliveira
\rule{\linewidth}{1.pt}

\begin{center}
	\Large\textbf{Exerc�cio EAD, 04/09/2018} 
\end{center}

\section*{Dedu��o Natural}

\subsection*{Quest�o 1.} Prove os seguintes sequentes usando dedu��o natural:

\begin{enumerate}[(a)]
	\item $\neg (\neg p \lor q) \vdash p$.
\end{enumerate}

{\color{blue}
	\begin{proof}{60pt}{80pt}
		\state{$\neg (\neg p \lor q)$}{PREMISSA}
		\assumption{3}{
		\state{$\neg p$}{HIPOTESE}
		\state{$\neg p \lor q$}{$\lor_i$ 2}
		\state{$\bot$}{$\neg_e$1,3}
		}
		\state{$\neg \neg p$}{$\neg_e$ 2-4}
		\state{$p$}{$\neg \neg_e$ 5}
	\end{proof}
}

\begin{enumerate}[(b)]
	\item $p \land \neg q \to r, \neg r, p \vdash q$.
\end{enumerate}
{\color{blue}
	\begin{proof}{60pt}{80pt}
		\state {$p \land q \to r$}{PREMISSA}
		\state{$\neg r$}{PREMISSA}
		\state{$p$}{PREMISSA}
		\assumption{4}{
	    \state{$\neg q$}{HIPOTESE}
	    \state{$p \land \neg q$}{$\land_i$ 3,4}
	    \state{$r$}{$\to_e$ 1,5}
	    \state{$\bot$}{$\neg_e$ 2,6}		
    }
    \state{$q$}{$\neg_i$ 4-7}
	\end{proof}
}

\begin{enumerate}[(c)]
	\item $(p \lor q) \lor r \vdash p \lor (q \lor r)$.
\end{enumerate}
{\color{blue}
	\begin{proof}{60pt}{80pt}
		\state{$(p \lor q) \lor r$}{PREMISSA}
		\assumption{7}{
		  \state{$p \lor q$}{HIPOTESE}
		  \assumption{2}{
		    \state{$p$}{HIPOTESE}
		    \state{$p \lor (q \lor r)$}{$\lor_i$ 3}
		  }
		  \assumption{3}{
		    \state{$q$}{HIPOTESE}
		    \state{$q \lor r$}{$\lor_i$ 5}
		    \state{$p \lor (q \lor q)$}{$\lor_i$6}
		  }
		  \state{$p \lor(q \lor r)$}{$\lor_e$2, 3-4, 5-7}
		}
		\assumption{2}{
		  \state{$r$}{HIPOTESE}
		  \state{$(p \lor q) \lor r$}{$\lor_i$ 9}
		}
		\state{$(p \lor q) \lor r$}{$\lor_e$2, 2-8, 9-10}
	\end{proof}
}

\begin{enumerate}[(d)]
	\item $(p \land q) \lor (p \land r) \vdash p \land (q \lor r)$.
\end{enumerate}
{\color{blue}
	\begin{proof}{60pt}{80pt}
	\state{$(p \land q) \lor (p \land r)$}{PREMISSA}
	\assumption{2}{
	  \state{$p \land q$}{HIPOTESE}
	  \state{$p$}{$\land_e$1,2}
	}
	\assumption{2}{
	  \state{$p \land r$}{HIPOTESE}
	  \state{$p$}{$\land_e$1,4}
	}
	\state{$p$}{$\lor_e$1, 2-3, 4-5}
	\assumption{3}{
	  \state{$p \land q$}{HIPOTESE}
	  \state{$q$}{$\land$2, 7}
	  \state{$q \lor r$}{$\lor_i$8}
	}
	\assumption{3}{
	  \state{$p \land r$}{HIPOTESE}
	  \state{$r$}{$\land_e$2, 10}
	  \state{$q \lor r$}{$\lor_i$11}
	}
	\state{$q \lor r$}{$\lor_e$1, 7-9, 10-12}
	\state{$p \land (q \lor r)$}{$\land_i$6,13}
	\end{proof}
}

\end{document}
